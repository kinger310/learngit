\resheading{项目经历}
  \begin{itemize}[leftmargin=*]
    \item
      \ressubsingleline{Flink及Flink SQL开发}{}{2020.04 -- 至今}
      {\small
      \begin{itemize}
        \item 开发环境: Flink
        \item 项目属性:实时计算平台
        \item 项目目标: 推动Flink及Flink SQL新环境的使用
        \item 职责成果: 
        \begin{itemize}
          \item (提效)发布了Flink-1.13项目模板和相关基础配置,任务包从50M精简到1M,任务发布时间从60秒到30秒;解决同事在新环境中遇到的问题,推动Flink-1.13在生产上应用、落地
          \item (优化)解决反压问题;解决作业浪费资源的问题,通过实验说明单核多Slot具备生产能力,为公司节省近100万元的物理机成本;解决Flink作业日志耗尽硬盘空间的问题,生产作业日志实现滚动
          \item (推新)在公司开Flink SQL讲座;搭建Flink SQL平台,编写示例代码,推广使用Flink SQL开发;解决如Flink SQL连接Hive,Influx,Hudi等问题
        \end{itemize}
      \end{itemize}
      }
  \end{itemize}

  \begin{itemize}[leftmargin=*]
    \item
      \ressubsingleline{数据湖研究与落地}{}{2021.08 -- 2021.12}
      {\small
      \begin{itemize}
        \item 开发环境: Hudi, Flink, CDH
        \item 项目属性:实时数仓 -- 数据湖
        \item 项目目标: 从零到一搭建数据湖,使公司数据湖能力从无到有
        \item 职责成果: 
        \begin{itemize}
          \item (环境)独自负责技术攻关。以Hudi为平台,从零到一搭建数据湖。攻克难点如下:修改源码解决兼容性问题;解决Hudi同步Hive问题,压力测试等;最终提供了可靠的hudi环境
          \item (案例)仿真数据案例快速验证;压力测试;基于案例进行各项参数的验证;如控制小文件数量、自定义分区、bulk-insert功能等;实际上线了催收场景的案例,最终案例结果比对一致
          \item (资源)对内存、CPU、运行参数进行多次调试;最终生产任务运行稳定,资源开销合理,故障恢复能力强
        \end{itemize}
      \end{itemize}
      }
  \end{itemize}

  \begin{itemize}[leftmargin=*]
    \item
      \ressubsingleline{天玑(特征管理配置)平台}{}{2020.09 -- 2021.02}
      {\small
      \begin{itemize}
        \item 开发环境: Golang
        \item 项目属性:数据Web服务
        \item 项目目标: 将逻辑相对简单的600多个特征字段做成自动化配置,用户免代码上线特征,节省至少一个开发人力
        \item 职责成果: 
        \begin{itemize}
          \item (立项)与产品经理、前端和测试沟通,分析产品模块,讨论字段上线流程、审批流程
          \item (开发)负责管理平台的字段模块开发和运行平台字段模块的开发;难点包括双平台的消息同步、字段逻辑的配置与解析、运行平台的高性能等
          \item (结果)项目成功上线,能将日常工作中30\%的特征开发任务交给需求方和测试来做配置和上线;项目稳定运行率99.99\%
        \end{itemize}
      \end{itemize}
      }
  \end{itemize}

  \begin{itemize}[leftmargin=*]
    \item
      \ressubsingleline{短信模型耗时优化}{}{2019.09 -- 2019.11}
      {\small
      \begin{itemize}
        \item 开发环境: Python, Golang
        \item 项目属性:风控流程优化
        \item 项目目标: 该模型耗时过长(3分钟左右),成为整个风控流程的瓶颈;优化耗时到5秒以内,同时保证模型结果一致
        \item 职责成果: 
        \begin{itemize}
          \item (分析)模型耗时分析,提炼优化点;发现瓶颈在于结巴分词,耗时占80\%
          \item (方案)将分词过程独立出来,做成Golang微服务,优化时间;同时重写分词算法,保证结果一致
          \item (结果)最终使得模型耗时降到10秒以内;同时保证结果一致率达到99.96\%
        \end{itemize}
      \end{itemize}
      }
  \end{itemize}
  % \begin{itemize}[leftmargin=*]
  %   \item
  %     \ressubsingleline{菜鸟网络仓储坪效优化项目}{高坪效算法开发与搭建}{2016.05 -- 2016.08}
  %     {\small
  %     \begin{itemize}
  %       \item 开发环境: Eclipse, MongoDB
  %       \item 项目属性:电子商务
  %       \item 项目简介: 仓储坪效即单位面积所容纳的货品件数。在电商大促、仓储资源有限的情况下,提升仓储坪效对缓解爆仓问题有重要意义
  %       \item 职责成果: 弄清项目需求和目标、分析海量仓储数据,设计了一个贪心算法来提升坪效,承担了项目程序代码的全部编写工作;实际在仓储单位应用实施项目,坪效提升了一倍
        
  %     \end{itemize}
  %     }
  % \end{itemize}
